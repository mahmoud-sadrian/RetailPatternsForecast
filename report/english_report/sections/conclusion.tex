\section{Conclusion}

This project provided a comprehensive exploration of both cross-sectional and temporal 
patterns in a real-world retail transaction dataset. Through meticulous preprocessing, 
detailed exploratory analysis, and the implementation of core machine learning concepts 
from first principles, several meaningful insights were uncovered about customer purchasing 
behavior and revenue dynamics.

The exploratory data analysis revealed strong structural characteristics inherent to retail 
environments. Product quantities and unit prices exhibited highly skewed distributions, 
with the majority of purchases involving low-cost items and small quantities. Sales were 
heavily concentrated among a limited set of popular products, reflecting a pronounced 
long-tail effect. Likewise, revenue was overwhelmingly dominated by the United Kingdom, 
highlighting an imbalance in geographic representation that must be considered when drawing 
business-level conclusions.

Manual association rule mining offered a deeper understanding of item co-occurrence patterns. 
By manually computing support, confidence, and lift—restricted to one-item and two-item 
frequent itemsets for computational tractability—we identified several strong purchasing 
associations with lift values significantly exceeding what would be expected under 
independence. The concentration of rules around popular products aligns with the previously 
observed long-tail distribution and confirms consistent customer behavior in the co-purchasing 
of specific decorative and household items.

The time series analysis further complemented these findings by revealing the temporal 
structure of daily revenue. Despite high levels of short-term volatility, the series 
displayed meaningful long-term trends, such as seasonal increases toward the end of the year. 
The manually implemented Simple Exponential Smoothing (SES) model demonstrated clear 
improvements over the naive forecasting approach, achieving lower prediction error and 
providing a stable estimate of expected revenue. Although SES cannot capture high-frequency 
oscillations, its simplicity and interpretability make it an effective tool for establishing 
baseline forecasts in retail environments.

Collectively, these analyses highlight the value of combining basket-level co-occurrence 
mining with time series forecasting to obtain a holistic understanding of retail behavior. 
The manual implementation of all core algorithms—Apriori components, rule evaluation metrics, 
and SES smoothing—reinforced fundamental data mining concepts and ensured transparency in 
modeling. Future work may extend this analysis by incorporating higher-order itemsets, 
FP-growth for enhanced scalability, seasonality-aware forecasting models such as Holt-Winters, 
or advanced deep learning architectures capable of capturing nonlinear demand patterns.

Overall, this project demonstrates how foundational analytical methods, when carefully 
applied and interpreted, can yield actionable insights into customer behavior, product 
relationships, and temporal revenue trends within retail data.