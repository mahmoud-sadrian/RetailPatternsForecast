\section{Preprocessing}

Before conducting association rule mining and time series forecasting, the raw dataset 
undergoes several essential preprocessing steps. These operations ensure that the resulting 
analyses are based on clean, consistent, and meaningful records.

\subsection{Data Cleaning}

The dataset contains irregularities that must be addressed. The following cleaning steps 
were applied:

\begin{itemize}
	\item Rows with missing \textbf{Description} were removed,
	\item Rows with missing \textbf{CustomerID} were discarded to avoid grouping ambiguity,
	\item Transactions with nonpositive \textbf{Quantity} or \textbf{UnitPrice} were excluded,
	\item A new monetary value field was computed:
	\[
	\text{TotalAmount} = \text{Quantity} \times \text{UnitPrice}.
	\]
\end{itemize}

These steps ensure that only valid and interpretable transactions are retained for further analysis.

\subsection{Basket Matrix Construction}

To prepare the dataset for association rule mining, a binary basket matrix was constructed. 
Each row corresponds to an invoice, and each column represents a product. The value is defined as:

\[
B_{ij} =
\begin{cases}
	1 & \text{if invoice } i \text{ contains product } j,\\
	0 & \text{otherwise}.
\end{cases}
\]

To maintain computational tractability for the manual Apriori implementation, only the 
top 50 most frequently purchased items were retained. This reduction preserves the most 
informative co-occurrence patterns while avoiding excessive dimensionality.

\subsection{Daily Revenue Time Series}

For forecasting purposes, total daily revenue was computed by aggregating monetary amounts per day:

\[
R(t) = \sum_{i \in \text{day } t} \text{TotalAmount}_i.
\]

This daily revenue series forms the foundation for the subsequent trend analysis and the 
manually implemented SES forecasting model.