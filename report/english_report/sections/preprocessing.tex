\section{Preprocessing}

Preprocessing consists of three major steps: cleaning, basket transformation, 
and time series construction.

\subsection{Data Cleaning}

Invalid rows were removed based on:
\begin{itemize}
	\item missing \textbf{Description},
	\item missing \textbf{CustomerID},
	\item nonpositive \textbf{Quantity} or \textbf{UnitPrice}.
\end{itemize}

The cleaned dataset ensures meaningful transactions and valid revenue values.

\subsection{Basket Matrix Construction}

A binary matrix \( B \) was created:
\[
B_{ij} =
\begin{cases}
	1 & \text{if invoice } i \text{ contains product } j,\\
	0 & \text{otherwise}.
\end{cases}
\]

To maintain computational feasibility in the manual Apriori implementation, 
only the top 50 most frequent products were retained.

\subsection{Daily Revenue Series}

Revenue for each day was defined as:
\[
R(t) = \sum_{i \in \text{transactions on } t} \text{TotalAmount}_i.
\]

This series forms the basis for SES forecasting.