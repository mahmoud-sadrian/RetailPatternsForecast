\section{Association Rule Mining (Manual Implementation)}

This section presents a detailed analysis of product co-occurrence patterns using manually 
implemented association rule mining. All components of the Apriori procedure—including 
support calculation, confidence, lift, and frequent itemset generation—were implemented 
without external libraries. The analysis focuses on 1-item and 2-item itemsets, which align 
well with the structure of the dataset, given the relatively small basket sizes observed.

% ---------------- Basket Size Distribution ----------------
\subsection{Basket Size Distribution}

\begin{figure}[H]
	\centering
	\includegraphics[width=0.85\textwidth]{basket_size_distribution}
	\caption{Distribution of the number of products per invoice.}
\end{figure}

The basket size distribution reveals that most invoices contain only one or two items, with 
frequency decreasing rapidly as basket size increases. This exponential decay pattern is 
typical in retail datasets, where customers tend to purchase only a few small, inexpensive 
items per transaction.

Only a very small fraction of invoices contain more than ten products, and baskets larger than 
fifteen items are extremely rare. This supports the methodological choice of restricting 
frequent itemset mining to 1-item and 2-item combinations, as higher-order itemsets would 
lack sufficient support and contribute little to the analysis.

% ---------------- Top 15 Frequent Items ----------------
\subsection{Most Frequently Purchased Items}

\begin{figure}[H]
	\centering
	\includegraphics[width=0.95\textwidth]{top15_items}
	\caption{Top 15 most frequently purchased products.}
\end{figure}

The most frequently purchased items are predominantly low-cost decorative or gift products 
such as candle holders, cake stands, lunch bags, and bunting accessories. The item 
\textit{WHITE HANGING HEART T-LIGHT HOLDER} stands out significantly, appearing in nearly 2{,}000 
invoices.

This high concentration of sales among a small set of items reflects a long-tail distribution, 
where a handful of products contribute disproportionately to overall transactions. As a 
consequence, association rules tend to form around these popular items rather than the more 
sparsely purchased products.

% ---------------- Support vs Confidence (Manual Rules) ----------------
\subsection{Support--Confidence--Lift Analysis}

\begin{figure}[H]
	\centering
	\includegraphics[width=0.85\textwidth]{rules_scatter}
	\caption{Support vs.\ confidence for manually generated rules, colored by lift.}
\end{figure}

The scatter plot illustrates the distribution of association rules by their support and 
confidence values, with lift represented through a color gradient. Several key patterns 
emerge from this visualization:

\begin{itemize}
	\item \textbf{Support values} range from approximately 0.02 to 0.04, indicating that the 
	rules are based on sufficiently frequent item co-occurrences.
	
	\item \textbf{Confidence values} vary between 0.30 and 0.68, demonstrating that several 
	rules provide moderately strong predictive power.
	
	\item \textbf{Lift values} span from roughly 3 to above 10. A lift greater than 3 already 
	signals strong positive association, while lift values exceeding 7--10 suggest highly 
	non-random co-purchasing behavior.
	
	\item Multiple clusters are visible in the support--confidence plane, indicating that 
	different item pairs form distinctive behavioral patterns rather than a single uniform 
	trend.
\end{itemize}

Overall, the plot confirms that the manually implemented rule mining procedure successfully 
captured meaningful relationships in the dataset. The presence of rules with both high 
confidence and high lift highlights particularly strong purchasing associations worthy of 
further exploration in retail analytics contexts.