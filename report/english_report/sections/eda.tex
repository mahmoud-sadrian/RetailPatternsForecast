\section{Exploratory Data Analysis}

This section provides a descriptive overview of the dataset through several visualizations, 
highlighting key statistical properties of product quantities, unit prices, country-level revenue 
distribution, and product popularity. These patterns help reveal inherent characteristics of the 
retail environment and guide subsequent association rule mining.

% ---------------- Quantity Distribution ----------------
\subsection{Quantity Distribution}

\begin{figure}[H]
	\centering
	\includegraphics[width=0.85\textwidth]{quantity_distribution}
	\caption{Distribution of purchased quantities.}
\end{figure}

The distribution of quantities is extremely right-skewed. Almost all transactions contain 
very small purchase quantities—typically between 1 and 3 units—resulting in a tall, narrow bar 
at the left end of the distribution. A very small number of transactions include exceptionally 
large quantities (up to tens of thousands of units), forming extreme outliers that are barely 
visible in the histogram due to the overwhelming mass at low values.

Such a distribution is characteristic of retail datasets, where customers primarily buy 
individual items or small bundles, while very large purchases may represent bulk orders, 
restocking events, or data-entry anomalies.

% ---------------- UnitPrice Distribution ----------------
\subsection{UnitPrice Distribution}

\begin{figure}[H]
	\centering
	\includegraphics[width=0.85\textwidth]{unitprice_distribution}
	\caption{Distribution of unit prices (log scale).}
\end{figure}

Unit prices span several orders of magnitude, necessitating the use of a logarithmic x-axis. 
Most products fall within a relatively low price range (roughly 0.1 to 10 units), while a few 
items exhibit significantly higher prices. These rare but high-value items create a long right tail.

The wide spread in price values indicates the need for proper scaling in downstream analysis, 
especially when measuring similarity or distance across transactions.

% ---------------- Top 10 Products ----------------
\subsection{Top 10 Products by Invoice Count}

\begin{figure}[H]
	\centering
	\includegraphics[width=0.95\textwidth]{top10_products}
	\caption{Most frequently purchased products.}
\end{figure}

The top-selling products are dominated by inexpensive decorative or gift items. Notably, the item 
\textit{WHITE HANGING HEART T-LIGHT HOLDER} appears significantly more frequently than all other 
products, followed by several bags, ornaments, and household accessories.

This strong concentration of sales among a few items illustrates a classic long-tail structure: 
a small set of popular products accounts for a disproportionately large share of transactions. 
These products play a central role in the formation of frequent itemsets and association rules.

% ---------------- Revenue by Country ----------------
\subsection{Revenue by Country}

\begin{figure}[H]
	\centering
	\includegraphics[width=0.95\textwidth]{revenue_by_country}
	\caption{Total revenue aggregated by country.}
\end{figure}

Revenue is overwhelmingly dominated by the United Kingdom, which contributes more than 7 million 
units of total sales. All other countries contribute only small fractions in comparison, with a 
steep drop-off visible across the distribution. This imbalance suggests that the dataset is 
effectively single-country for analytical purposes, as more than 90\% of all activity originates 
from the UK.

Such extreme skewness does not negatively impact association rule mining but underscores the need 
for caution when interpreting geographic trends, as the dataset is not globally representative.