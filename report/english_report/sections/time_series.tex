\section{Time Series Analysis and Manual SES Forecasting}

This section presents the construction and analysis of a daily revenue time series, followed by 
a fully manual implementation of the Simple Exponential Smoothing (SES) forecasting method. 
The objective is to understand temporal purchasing patterns and compare the performance of 
a naive forecast with that of SES.

% ---------------- Daily Revenue ----------------
\subsection{Daily Revenue Overview}

\begin{figure}[H]
	\centering
	\includegraphics[width=0.85\textwidth]{daily_revenue}
	\caption{Daily total revenue over the full observation period.}
\end{figure}

The daily revenue series exhibits substantial day-to-day variability, including numerous sharp 
peaks and drops. Such volatility is characteristic of retail datasets, where fluctuations arise 
from irregular purchasing cycles, bulk orders, and seasonality. Despite the noise, a gradual 
upward trend emerges toward the end of the year, suggesting strengthening demand.

% ---------------- Moving Average ----------------
\subsection{Smoothed Trend Using Moving Average}

\begin{figure}[H]
	\centering
	\includegraphics[width=0.85\textwidth]{daily_revenue_ma7}
	\caption{Daily revenue with a 7-day moving average.}
\end{figure}

A 7-day moving average is applied to reduce high-frequency noise. The smoothed curve reveals more 
distinct seasonal behavior: an initial decline early in the year, a relatively stable middle period, 
and a noticeable upward trend approaching the final months. Several pronounced peaks likely correspond 
to promotional events or holiday periods.

% ---------------- Train/Test & SES ----------------
\subsection{SES Forecasting on Train/Test Split}

\begin{figure}[H]
	\centering
	\includegraphics[width=0.85\textwidth]{ses_vs_actual}
	\caption{Actual vs.\ SES forecast on the test interval.}
\end{figure}

The SES model produces a constant forecast across the test window, reflecting the fundamental SES 
formulation in which future predictions equal the estimated level of the series:
\[
\hat{y}_{t+h} = s_t.
\]
While the actual test data show significant fluctuations, the SES forecast provides a stable baseline 
that captures the central tendency of the revenue pattern more effectively than the naive method.

% ---------------- RMSE Comparison ----------------
\subsection{Evaluation: Naive vs SES}

\begin{figure}[H]
	\centering
	\includegraphics[width=0.7\textwidth]{rmse_comparison}
	\caption{RMSE comparison between naive forecasting and manual SES.}
\end{figure}

The SES model achieves a lower RMSE relative to the naive benchmark, indicating improved predictive 
accuracy. The naive method—simply carrying forward the previous day's revenue—is particularly 
vulnerable to the high volatility of this dataset. In contrast, SES smooths out transient spikes and 
better estimates the underlying revenue level.

% ---------------- Forecasting ----------------
\subsection{14-Day SES Forecast}

\begin{figure}[H]
	\centering
	\includegraphics[width=0.85\textwidth]{ses_14day_forecast}
	\caption{Manual SES forecast for the next 14 days.}
\end{figure}

The 14-day forecast generated by the manual SES model is constant, reflecting the final estimated level 
of the historical series. Because revenue shows increased amplitude toward the end of the year, the SES 
forecast lies above the long-term mean. Although this simple model cannot capture short-term oscillations, 
it provides a stable and interpretable estimate of expected revenue.